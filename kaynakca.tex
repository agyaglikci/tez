\newpage
\pagestyle{plain}
\addcontentsline{toc}{chapter}{\numberline{KAYNAKLAR}}


\begin{thebibliography}{99}

\bibitem{cudaProgrammingStructure} Lippert, A. (2009). NVIDIA GPU Architecture for General Purpose Computing, 18.

\bibitem{dspArchitectures} Edwin. J. Tan, Wendi. B. Heinzelman. 2003. DSP Architectures: Past, Present and Futures. ACM Sigarch Computer Architecture News

\bibitem{hallmans2013gpgpu} Hallmans, Daniel, et al. 2013. GPGPU for industrial control systems. IEEE 18th Conference on Emerging Technologies \& Factory Automation ETFA

\bibitem{Kilgariff2005} Emmett Kilgariff and Randima Fernando. 2005. The GeForce 6 series GPU architecture. In ACM SIGGRAPH 2005 Courses SIGGRAPH '05, John Fujii (Ed.). ACM, New York, NY, USA

\bibitem{kirk2007nvidia} Kirk, D. 2007. NVIDIA CUDA software and GPU parallel computing architecture. ISMM Vol. 7, pp. 103-104

\bibitem{stone2010opencl} Stone, J. E., Gohara, D., \& Shi, G. 2010. OpenCL: A parallel programming standard for heterogeneous computing systems. Computing in science \& engineering, 12(3), 66

\bibitem{kuon2007measuring} Kuon, I., \& Rose, J. 2007. Measuring the gap between FPGAs and ASICs. Computer-Aided Design of Integrated Circuits and Systems, IEEE Transactions on, 26(2), 203-215.

\bibitem{smith1997matlab}Smith, R.L., \"The MATLAB project book for linear algebra\", 1997 Prentice Hall

\bibitem{eigenvalueComputation} Gotze, J.; Paul, S.; Sauer, M., \"An efficient Jacobi-like algorithm for parallel eigenvalue computation,\" Computers, IEEE Transactions on , vol.42, no.9, pp.1058,1065, Sep 1993

\bibitem{flynnTaxonomy}Flynn, M. J. (September 1972). \"Some Computer Organizations and Their Effectiveness\". IEEE Trans. Comput. C–21 (9): 948–960. doi:10.1109/TC.1972.5009071

\bibitem{shivakumar2002modeling} Shivakumar, P., Kistler, M., Keckler, S. W., Burger, D., \& Alvisi, L. (2002). Modeling the effect of technology trends on the soft error rate of combinational logic. In Dependable Systems and Networks, 2002. DSN 2002. Proceedings. International Conference on (pp. 389-398). IEEE.

\bibitem{seiler2008larrabee} Seiler, L., Carmean, D., Sprangle, E., Forsyth, T., Abrash, M., Dubey, P., ... \& Hanrahan, P. (2008). Larrabee: a many-core x86 architecture for visual computing. ACM Transactions on Graphics (TOG), 27(3), 18.

\bibitem{molka2009memory} Molka, D., Hackenberg, D., Schone, R., \& Muller, M. S. (2009, September). Memory performance and cache coherency effects on an Intel Nehalem multiprocessor system. In Parallel Architectures and Compilation Techniques, 2009. PACT'09. 18th International Conference on (pp. 261-270). IEEE

\bibitem{hackenberg2009comparing} Hackenberg, D., Molka, D., \& Nagel, W. E. (2009, December). Comparing cache architectures and coherency protocols on x86-64 multicore SMP systems. InProceedings of the 42Nd Annual IEEE/ACM International Symposium on microarchitecture (pp. 413-422). ACM

\bibitem{MCSE.2012.23} Heinecke, A., Klemm, M., Bungartz H.J., \"From GPGPU to Many-Core: Nvidia Fermi and Intel Many Integrated Core Architecture\" Computing in Science and Engineering, vol. 14, no. 2, pp. 78-83, March-April, 2012 

\bibitem{cpuGpuMemoryTable} http://supercomputingblog.com/cuda/cuda-memory-and-cache-architecture/

\bibitem{tileArchitecture} Wentzlaff, D., Griffin, P., Hoffmann, H., Bao, L., Edwards, B., Ramey, C., Mattina, M., Miao, C.-C., III, J. F. B. \& Agarwal, A. (2007). On-Chip Interconnection Architecture of the Tile Processor.. IEEE Micro, 27, 15-31. 

\bibitem{nvidiaPTXISA} http://docs.nvidia.com/cuda/parallel-thread-execution/\#texture-instructions

\bibitem{x86ISA} http://en.wikipedia.org/wiki/X86\_instruction\_listings

\bibitem{MIPSISA} http://www.mrc.uidaho.edu/mrc/people/jff/digital/MIPSir.html

\bibitem{superscalar600mhz} Gieseke, B. A., Allmon, R. L., Bailey, D. W., Benschneider, B. J., Britton, S. M., Clouser, J. D., ... \& Wilcox, K. E. (1997, February). A 600 MHz superscalar RISC microprocessor with out-of-order execution. In Solid-State Circuits Conference, 1997. Digest of Technical Papers. 43rd ISSCC., 1997 IEEE International (pp. 176-177). IEEE.

\bibitem{superscalarpatent}Garg, S., Hagiwara, Y., Lau, T. L., Lentz, D. J., Miyayama, Y., Trang, Q. H., ... \& Wang, J. (1996). U.S. Patent No. 5,560,032. Washington, DC: U.S. Patent and Trademark Office.

\bibitem{interleavedMultithreading} Laudon, J., Gupta, A., \& Horowitz, M. (1994). Interleaving: A multithreading technique targeting multiprocessors and workstations. ACM SIGPLAN Notices, 29(11), 308-318.

\bibitem{Controldatacorp}Control Data Corp, «CDC Cyber 170 Computer Systems; Models 720, 730, 750, and 760; Model 176 (Level B); CPU Instruction Set; PPU Instruction Set,» pp. 2-44.

\bibitem{TeraMTA}A. e. a. Snavely, \"Multi-processor Performance on the Tera MTA,\" in IEEE Computer Society Proceedings of the 1998 ACM/IEEE conference on Supercomputing, 1998. 


\end{thebibliography}