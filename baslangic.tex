\documentclass{etutez}
\usepackage[toc,page,titletoc]{appendix}
\usepackage[T1]{fontenc}
\usepackage[utf8]{inputenc}
\usepackage[turkish]{babel}
\usepackage{titlesec}
\usepackage{graphicx}
\usepackage{tabularx}
%\usepackage{longtable}
%\usepackage{mathtools}
%\usepackage[export]{adjustbox}


\pagenumbering{roman}

\thesistype{M.Sc.} %Yuksek lisans icin M.Sc. doktora icin Ph.D.
\teztipi{Yüksek Lisans}  %Tez tipini yazin

\keywords{FPGA, accelerator, co-processor, OpenCL}
\anahtarsoz{FPGA, hızlandırıcı, yardımcı işlemci, OpenCL}

\title{TITLE OF THE THESIS}
\baslik{FPGA TABANLI SAYISAL S\.{I}NYAL \.{I}\c{S}LEME ALGOR\.{I}TMALARINA \"{O}ZELLE\c{S}T\.{I}R\.{I}LM\.{I}\c{S} YARDIMCI \.{I}\c{S}LEMC\.{I} TASARIMI}  % buyuk harflerle yazilmali

\yazar{ABDULLAH G\.{I}RAY YA\u{G}LIK\c{C}I}    % buyuk harflerle yazilmali
\yazarkucuk{Abdullah Giray Yağlıkçı}    % Soyadi Buyuk harflerle yazilmali
\enstitu{FEN B\.{I}L\.{I}MLER\.{I} } % buyuk harflerle yazilmali
\enstitukucuk{Fen Bilimleri } % kucuk harflerle yazilmali
\institute{Institute of Natural and Applied Sciences}
\bolum{B\.{I}LG\.{I}SAYAR M\"UHEND\.{I}SL\.{I}\u{G}\.{I} } % buyuk harflerle yazilmali
\bolumkucuk{Bilgisayar Mühendisliği } % kucuk harflerle yazilmali
\dept{Computer Engineering}
\supervisor{Assoc. Prof. Oğuz ERG\.{I}N}
\tezyoneticisi{Doç. Dr. Oğuz ERG\.{I}N}
\juribaskani{....}
\juriuyesi{....}
\anablmdalibsk{Doç. Dr. Erdoğan Doğdu}
\enstitumuduru{Prof. Dr. \"Unver KAYNAK}
%\copyrightyear{2001}
\submitdate{August 2014}  % buyuk harflerle yazilmali
\tarih{A\u{G}USTOS 2014}   % buyuk harflerle yazilmali
\tarihkucuk{Ağustos 2014}   % kucuk harflerle yazilmali


\newtheorem{thm}{Theorem}
\newtheorem{preexam}{Example}
\newenvironment{exam} {\begin{preexam}\rm}{\end{preexam}}
\newtheorem{lem}{Lemma}
\newtheorem{proprty}{Property}
\newtheorem{cor}{Corollary}
\newtheorem{defn}{Definition}
\newcommand{\pe}{\preceq}
\newcommand{\po}{\prec}




\hyphenpenalty=5000
\tolerance=1000

\begin{document}


\titlepageMS   % Yuksek lisans tezi kapak sayfasi
%\titlepagePhD    % Doktora tezi kapak sayfasi
%\setcounter{page}{2}
\signaturepageMS  % Yuksek lisans tezi imza sayfasi
%\signaturepagePhD     % Doktora tezi imza sayfasi
\tezbildirimsayfasi    % tez bildirim sayfasi


\begin{ozet}
	Sayısal sinyal işlemede yaygın olarak kullanılan fonksiyonların büyük bir veri seti üzerinde çalıştırılması durumunda paralelleştirilmesi, yürütme zamanını kritik bir şekilde azaltmaktadır. Farklı veriler üzerinde aynı işlemlerin tekrarlandığı algoritmalarda performans artışı sağlamak adına iş parçalarının paralel yürütülebilmesi için çok çekirdekli işlemciler, GPGPU, ASIC tasarımlar ve FPGA tabanlı sistemler algoritmanın koşturulacağı platformların başında gelir. Her bir platformun kendi avantajları ve dezavantajları olmakla beraber, düşük maliyet ile yüksek paralellik sağladığı için GPGPU ve FPGA'ler son yıllarda en yaygın kullanılan platformlardır. Bu tez, ASELSAN - TOBB ETÜ iş birliğinde yürütülen, çıktısı FPGA tabanlı ve OpenCL destekli, ölçeklenebilir ve özelleştirilebilir tasarıma sahip bir yardımcı işlemci ünitesi olan projenin donanım tasarımı kısmını kapsar. Tez çalışmalarına paralel olarak derleyici tasarımı yapılmış fakat tez içeriğine dahil edilmemiştir.
\end{ozet}



\begin{abstract}
	Typical digital signal processing algorithms executes the same DSP functions on different data sets. Parallelizing this process dramatically decreases execution time of such kind of functions. There are 4 popular platforms for parallelized applications: Many-core processors, GPGPUs, ASIC chips and FPGA based applications. Although each kind of platform has own pros and cons, GPGPU and FPGA based applications are more popular than others because of lower price and higher parallel processing capabilities. This MSc thesis consists of hardware design of a project which is managed by ASELSAN and TOBB ETÜ and the output of project is FPGA based OpenCL ready highly scalable and configurable co-processor. Although compiler works are in progress, this thesis only includes the harware design of co-processor.  
\end{abstract}


\begin{tesekkur}
 Bu çalışmayı tamamlamamda emeği geçen değerli danışman hocam Doç. Dr. Oğuz Ergin'e; kıymetli çalışma arkadaşlarım Hasan Hassan, Hakkı Doğaner Sümerkan, Serdar Zafer Can, Serhat Gesoğlu, Volkan Keleş ve Osman Seçkin Şimşek'e; tez çalışmam sırasında beni destekleyen aileme ve değerli arkadaşlarım Fahrettin Koç, Tuna Çağlar Gümüş ve Emrah İşlek'e; projeye desteğinden ötürü ASELSAN'a ve çalışma ortamımızı sağladığı için TOBB ETÜ Mühendislik Fakültesi ve Fen Bilimleri Enstitüsüne teşekkür ederim.
\end{tesekkur}



\pagestyle{plain}




\makeatother


