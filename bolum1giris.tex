\chapter{G\.{I}R\.{I}\c{S}} \label{chapter:giris}
Sayısal sinyal işleme algoritmalarında sıklıkla aynı işlem, farklı veriler üzerinde uygulanmaktadır. Geleneksel işlemcilerde bu tarz bir uygulama her veri için işlemin peşpeşe tekrarlanması ile gerçeklenir. Oysa ki algoritmaların bu özelliği, farklı veriler için uygulanacak aynı işlemin sırayla değil paralel çalıştırılması ile kayda değer performans artışlarını beraberinde getirir. Örneğin N elemanlı iki vektörün skalar çarpımı, N adet çarpma işleminden ve ardından N adet verinin toplanmasından oluşur. N adet çarpma işleminden herhangi birinin bir diğerini beklemeye ihtiyacı yoktur. Bu çarpma işlemlerinin peşi sıra yapıldığı ve paralel yapıldığı durumlar karşılaştırıldığında, paralel olan yöntemde N kata yakın performans artışı gözlenir.Paralelleştirmenin azımsanamayacak performans avantajından dolayı paralel çalışmayı destekleyecek donanım tasarımları üzerinde pek çok çalışma yapılmıştır. Literatürde öne çıkan çalışmaları 4 başlık altında toplamak mümkündür.  \par

Geleneksel işlemcilerde birden fazla iş parçacığının eş zamanlı çalıştırılabilmesi için çok çekirdekli mimari tasarımları yaygın olarak kullanılmaktadır. Çok çekirdekli işlemcilerde bir çekirdek üzerinde 1 veya daha fazla thread koşturulması ile sinyal işleme fonksiyonlarında paralellik sağlanmaktadır. Endüstriyel uygulamalarda kullanılan DSP(Digital Signal Processor) yongaları da çok çekirdekli işlemci mimarisine sahip özelleştirilmiş donanımlardır.\cite{dspArchitectures} Bu tarz mimarilerde çekirdeklerin programlanabilir olması uygulamada esneklik sağlar. Genel amaçlı çok çekirdekli CPU işlemciler, sinyal işleme uygulamalarında alternatiflerine göre daha az paralel ve daha yavaş kalırlarken DSP yongaları, ilave bir donanım olarak son ürünün ömrünü kısaltmakta ve güncellenebilirliğini azaltmaktadır.\cite{hallmans2013gpgpu} \par

Bilgisayar ekranına basılacak piksellerin renk ve parlaklık değerlerinin hızlı ve paralel bir biçimde hesaplanabilmesi için geliştirilen grafik işlemcileri çok sayıda çekirdeğe sahiptir.\cite{Kilgariff2005} Hemen her bilgisayarda bulunan grafik işlemcilerinin genel amaçlı paralel hesaplama gerektiren işlerde kullanılması ekonomik ve yüksek performasnlı bir çözüm olarak kendini göstermiştir. Grafik işlemcilerinin genel amaçlı kullanımını destekleyen iki kutup olarak NVidia ve Khronos grubu, sırasıyla CUDA ve OpenCL desteği sağlayarak GPGPU (General Purpose Graphics Processing Unit) kullanımını yaygınlaştırmıştır. \cite{kirk2007nvidia} \cite{stone2010opencl} GPGPU programlama ile uygulamaların paralelleştirilmesi ek donanım gerektirmediği için ekonomik, çok sayıda çekirdekten oluşan donanımlar olduğu için yüksek derecede paralelleştirilebilir bir donanım alternatifidir. Ticari donanımlar olan grafik işlemcilerinin dezavantajı ise birinci önceliği piksel değeri hesaplayan çekirdeklerden oluşması ve çok özel amaçlı işlerde performans bakımından yetersiz kalmasıdır. Burada bahsi geçen yetersizlik buyruk kümesi tasarımı ile ilgilidir.\par

GPGPU ve DSP üzerinde donanımsal değişiklik yapmak mümkün değildir. Donanımsal değişiklik istenen durumlarda, donanım tasarımına müdahale edilebilen ASIC (Application Specific Integrated Circuit) tasarımlar ve FPGA(Field Programmable Gate Array) tabanlı sistemler ön plana çıkar. ASIC tasarımlar yarı iletken seviyesinde tasarlanan devrelerden oluşurken FPGA tabanlı sistemler, adından da anlaşılacağı üzere, FPGA yongalarında hazır bulunan LUT (Lookup Table), kapı, bellek vb. yapılar kullanılarak gerçeklenir. Her iki yaklaşımın diğerlerinden farkı yazılım seviyesinde değil donanım seviyesine yapılan özelleştirme ile performans artışının sağlanmasıdır. ASIC - FPGA karşılaştırmasında ASIC uygulamalar yarı iletken seviyesinde, FPGA uygulamalar ise daha üst seviyede yapılır. Dolayısıyla ASIC tasarımdan alınan performans artışına FPGA seviyesinde erişilmesi mümkün değildir. Öte yandan ASIC uygulamaların, üretim gerektirdiği için maliyeti fazla, güncellenebilirliği azdır. \cite{kuon2007measuring} \par

Bu tez, sayısal sinyal işleme algoritmalarında yaygın olarak kullanılan fonksiyonların paralel çalıştırılması için tasarlanan FPGA tabanlı bir sistemin donanım tasarımını içerir. Söz konusu sistem ASELSAN ve TOBB ETÜ'nün ortak projesi olup, ASELSAN tarafından sayısal sinyal işleme uygulamalarında kullanılması planlanmaktadır. Dolayısıyla tasarımın temelini oluşturan kriterler ve fonksiyon listesi ASELSAN tarafından belirlenmiştir. \par

Tezin 3. bölümünde ASELSAN tarafından belirlenen tasarım kriterleri ve fonksiyon listesi özetlenmiş ve tasarım öncesi sistem özellikleri belirlenmiştir. 4. bölümde benzer özellikteki mimariler sunulmuş, avantajları ve dezavantajları tartışılmıştır. 5. bölümde buyruk kümesi ve boru hattı tasarımı anlatılmış, 6. bölümde ise mimari tasarımı alt modüllere ayrılarak her bir modülün tasarımı açıklanmıştır. 7. bölümde sonuçların sunumu ile tez sonlandırılmıştır.  
