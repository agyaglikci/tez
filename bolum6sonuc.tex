\chapter{SONUÇ}
Sayısal sinyal işleme algoritmaları, DSP ve GPGPU platformlarında koşturulmaktadır. Uygulamalar, karakteristik SIMD özellikleri sayesinde paralelleştirilerek hızlandırılmaya oldukça elverişli olduğu için son yıllarda GPGPU platformlarında CUDA ve OpenCL kullanılarak gerçeklenen sinyal işleme uygulamaları türetilmiştir. GPGPU mimarileri donanım seviyesinde özelleştirilemezken, platform bağımsız OpenCL sayesinde yüksek seviyede esnekliğe sahiptir. Öte yandan bazı uygulamalarda donanım seviyesinde değişiklikler yapmak istenebilir. Donanım seviyesinde değişiklik GPGPU donanımlarında mümkün olmadığı gibi ASIC tasarımlarda da maliyetlidir. Bu noktada FPGA tabanlı OpenCL destekli bir mimari hem donanım seviyesinde müdahale edilebilir, ölçeklenebilir bir yapıya hem de yazılım seviyesinde OpenCL'in sağladığı esnekliğe sahip olacaktır.  \par
Tez çalışması dahilinde tasarlanan FPGA tabanlı yardımcı işlemci ünitesi tümüyle ölçeklenebilir ve özelleştirilebilir bir yapıya sahip olarak tasarlanmıştır. Belirlenen buyruk kümesi OpenCL kullanılarak yazılmış herhangi bir uygulamayı koşturabilecek kabiliyete sahiptir. Boru hattı mimarisi, aralıklı işlem modeli ve yazmaç öbeği, farklı warplardan buyrukların bir arada çalıştırılması ile veri bağımlılıkları çözülmeksizin boru hattının etkin kullanımını sağlamaktadır.  \par
Tasarımın hiyerarşik olmasını sağlayan adalardan oluşan mimaride her ada içinde parametrik miktarda SIMD lane vardır. Bir adanın içindeki tüm SIMD lane'ler için aynı anda aynı buyruk çalıştırılırken farklı adalarda farklı buyruklar çalıştırılabilir. Bu sayede ortak kaynak kullanımı gerektiren ana bellek erişimi işlemlerine harcanan süre farklı adalar arasında faz farkı oluşturularak gizlenebilir. \par
Kaçınılmaz olarak yüksek miktarda zaman kaybına sebep olan bellek işlemleri için tasarlanan paylaşımlı bellek, 