\chapter{SONUÇ}
Sayısal sinyal işleme algoritmaları, DSP ve GPGPU platformlarında koşturulmaktadır. Uygulamalar, karakteristik SIMD özellikleri sayesinde paralelleştirilerek hızlandırılmaya oldukça elverişli olduğu için son yıllarda GPGPU platformlarında CUDA ve OpenCL kullanılarak gerçeklenen sinyal işleme uygulamaları türetilmiştir. GPGPU mimarileri donanım seviyesinde özelleştirilemezken, platform bağımsız OpenCL sayesinde yüksek seviyede esnekliğe sahiptir. Öte yandan bazı uygulamalarda donanım seviyesinde değişiklikler yapmak istenebilir. Donanım seviyesinde değişiklik GPGPU donanımlarında mümkün olmadığı gibi ASIC tasarımlarda da maliyetlidir. Bu noktada FPGA tabanlı OpenCL destekli bir mimari hem donanım seviyesinde müdahale edilebilir, ölçeklenebilir bir yapıya hem de yazılım seviyesinde OpenCL'in sağladığı esnekliğe sahip olacaktır. Bu motivasyon ile tez çalışması dahilinde tasarlanan FPGA tabanlı yardımcı işlemci ünitesi tümüyle ölçeklenebilir ve özelleştirilebilir bir yapıya sahip olarak tasarlanmıştır.\par
\begin{itemize}
\item Belirlenen buyruk kümesi OpenCL kullanılarak yazılmış herhangi bir uygulamayı koşturabilecek kabiliyete sahiptir. 
\item Boru hattı mimarisi, aralıklı işlem modeli ve yazmaç öbeği, farklı warplardan buyrukların bir arada çalıştırılması ile veri bağımlılıkları çözülmeksizin boru hattının etkin kullanımını sağlamaktadır.
\item Tasarımın hiyerarşik olmasını sağlayan adalardan oluşan mimaride her ada içinde parametrik miktarda SIMD lane vardır. Bir adanın içindeki tüm SIMD lane'ler için aynı anda aynı buyruk çalıştırılırken farklı adalarda farklı buyruklar çalıştırılabilir. Bu sayede ortak kaynak kullanımı gerektiren ana bellek erişimi işlemlerine harcanan süre farklı adalar arasında faz farkı oluşturularak gizlenebilir.
\item Threadler arası veri paylaşımı paylaşımlı bellek üzerinden sağlanır. Her adada bir paylaşımlı bellek bulunmaktadır.
\item Paylaşımlı bellek farklı block ram'lere dağıtılmış bir adres uzayı üzerinde işlem yapmaktadır. Bu sayede SIMD lane adet port üzerinden gelen istekler çoğu durumda eş zamanlı olarak cevaplanabilmektedir. 
\item Paylaşımlı bellek adres uzayı, block ram'lere dağıttılırken ardışık adresler farklı block ramler'de olacak şekilde soyutlama yapılmıştır. Farklı portlardan gelen isteklerin eş zamanlı çalıştırılabilmesi için bu soyutlama ile yazılım seviyesinde optimizasyon imkanı sağlanmıştır. 
\item Hiyerarşik yapı, yazılım tarafından bakıldığında OpenCL destekli diğer platformlar gibi bazı kısıtlar getirmektedir. OpenCL ile gerçeklenmiş çekirdekler thread bloklarından oluşur. Her thread bloğunun içindeki threadler arasında veri paylaşımına izin vardır. Tosun mimarisinde her bir ada içinde paylaşımlı bellek gerçeklendiğinden bir adada çalışan herhangi bir thread, aynı ada içinde çalışan başka herhangi bir thread ile veri paylaşımında bulunabilir. Mevcut mimaride thread bloğu içindeki en fazla thread sayısı $N_{SIMD lane} x N_{warp}$ şeklinde ifade edilebilir. 
\item Hesaplama modüllerinin sabitlenmiş giriş çıkış ara yüzlerine uygun olmak şartı ile herhangi bir özel hesaplama modülü daha sonra tasarıma ilave edilebilir. Mevcut mimaride belirtilen buyruk kümesindeki tüm işlemler için gerekli olan hesaplama modülleri değişik sayılarda boru hattının hesap aşamasına eklenmiştir. Daha sonra ilave edilmek istenen bir hesap biriminden istenilen adette aynı giriş çıkış standardına bağlı kalınarak hesap aşamasına eklenebilir. Böylece buyruk kümesi genişletilebilir. 
\end{itemize}   

\section{Tosun Performans Analizi}
Tasarlanan mimaride performansın bir ölçütü buyrukların kaç çevrimde tamamlandığıdır. Her buyruğun boru hattını tamamlama süresi belli olsa da bir uygulamanın çalışmasında boru hattının etkin kullanımına göre toplam süre değişiklik gösterir. Mimariye uygun yazılan bir program için en iyi durumda boru hattı bir kere doldurulduktan sonra her çevrimde bir buyruk tamamlanır. Boru hattının uzunluğu hesap aşaması haricinde tüm buyruklar için sabittir. Hesap aşamasında ise her işlemin farklı bir süresi vardır. Her bir buyruk için hesap aşamasının tamamlanma süresi Tablo \ref{table:hesapSureleri}'de sunulmuştur. Tablo \ref{table:hesapSureleri}'de verilen "Hesaplama Ç.S.", matematiksel işlem için kullanılan zamandır. Hesap aşamasında hesaplama modülüne bağlı olarak giriş ve çıkışta kuyruk yapıları kullanılır. Boru hattının etkin kullanımı için eklenen bu kuyruklar, çevrim sayısında artışa sebep olur. Kuyrukların etkisiyle beraber, boru hattının hesaplama aşaması için her bir buyruğun toplam çevrim sayıları da "Boru Hattı Aşaması Ç.S." sütununda verilmiştir. \par

\begin{longtable}{p{50pt} p{90pt} p{90pt}}
\caption{Her Bir Buyruk için Hesap Aşaması Süreleri} \label{table:hesapSureleri} \\
\textbf{Buyruk} & \textbf{Hesaplama Ç.S.} & \textbf{Boru Hattı Aşaması Ç.S.}\\ 
\hline 
\endfirsthead

\multicolumn{2}{c}%
{{\bfseries \tablename\ \thetable{} -- devam}} \\
\textbf{Buyruk} & \textbf{Hesaplama Ç.S.} & \textbf{Boru Hattı Aşaması Ç.S.}\\ 
\hline 
\endhead

\hline \multicolumn{2}{r}{\textbf{Sonraki sayfada devam etmektedir.}} \\ 
\endfoot

\hline \hline
\endlastfoot
  addi		&  2 	&  4 \\
  andi 		&  0  &  2 \\
  ori 		&  0  &  2 \\
  xori 		&  0  &  2 \\
  divi 		&  20 & 24 \\
  muli 		&  2  &  4 \\
  subi 		&  2  &  4 \\
  movi 		&  0  &  2 \\
  movhi		&  0  &  2 \\
  fabs  	&  0  &  2 \\
  fadd  	&  5  &  7 \\
  fcom  	&  1  &  3 \\
  fdiv  	&  10 & 14 \\
  fmul  	&  2  &  4 \\
  fsqrt  	&  14 & 18 \\
  fcos  	&  28 & 32 \\
  fsin  	&  28 & 32 \\
  ffma  	&  9  & 11 \\
  ffms  	&  9  & 11 \\
  fmin  	&  0  &  2 \\
  fmax  	&  0  &  2 \\
  fln 	 	&	 12 & 16 \\
  fmod 		&	 10 & 14 \\
  f2int 	&	    &  4 \\
  int2f		&	    &  4 \\
  fchs		&	  0 &  2 \\
  fexp 		&	  8 & 12 \\
  add 		&	  2 &  4 \\
  and 		&	  0 &  2 \\
  or 		  &	  0 &  2 \\
  xor		  &	  0 &  2 \\
  div  		&  20 & 24 \\
  mul  		&   2 &  4 \\
  shl  		&   1 &  3 \\
  shr  		&   1 &  3 \\
  shra  	&   1 &  3 \\
  sub  		&   2 &  4 \\
  min  		&   1 &  3 \\
  max  		&   1 &  3 \\
  chs  		&   2 &  4 \\
  not  		&   0 &  2 \\
  abs  		&   2 &  4 \\
  com  		&   1 &  3 \\
  mod  		&   2 & 24 \\
  brv 		&	  0 &  2 \\
  bfr 		&	  1 &  3 \\
  br 			&	  x &  x \\
  fin 		&	  x &  x \\
  ldshr 	&	 7-22 & 1 - 26 \\
  stshr 	&	 7-22 & 1 - 26 \\
  sync		&	  x  & x \\
  ldram 	&	   &  \\
  stram		&	   &  \\
  mov 		&    0 & 2 \\
  jmp  		&    x & x \\
  
\end{longtable}

Tosun üzerinde çalıştırılan bir buyruk işlenmek üzere bir adaya alındıktan sonra tüm boru hattı aşamalarından geçerek işlemini tamamlar. Tablo \ref{table:hesapSureleri}'de sunulan boru hattı aşaması çevrim sayıları yalnızca hesaplama aşamasına ait verilerdir. Nitekim buyruklar arası çevrim sayısı farklılıkları yalnızca hesaplama aşamasında oluşmaktadır. Diğer tüm boru hattı aşamaları, tüm buyruklar için sabit çevrim sayısına sahiptir.\par

Tasarlanan boru hattında bir nuyruğun çalışması warp seçimi ile başlar. Warp seçimi donanımda aktif warp'ların tutulduğu bir tablo üzerinde Round Robin algoritması ile seçim yapılmasından ibarettir ve 1 çevrimde sonuçlanır. \par 

Seçilen warp için sıradaki buyruk bellekten çekilir. Bu aşamada buyruk ön belleğinde söz konusu buyruk bulunursa, 1 çevrimde aşama geçilir. Eğer buyruk önbellekte yoksa, ana bellek üzerinden buyruğun çekilmesi gerekmektedir. Ana belleğin cevap süresi anlık yoğunluğa göre değişmektedir. En kötü durum, tüm adaların hem veri hem buyruk portlarından istek gelirken aynı zamanda PCI ve ana bellek arasında da veri akışı varken, hiçbir isteğin önbellekte bulunamaması durumudur. En kötü durum tahmini cevap süresi $35 x ( N_{ada} x 2 + 1 )$ şeklinde ifade edilebilir. \par

Buyruk çözme aşamasında yalnızca bit gruplarının ayrılması işlemi yapıldığından 1 çevrimde geçilebilir. Yazmaç çekme aşamasında her SIMD lane kendine ait yazmaç öbeğinden 1, 2 veya 3 adet yazmacın değerini okur. Yazmaç öbeğinin cevap süresi en kötü durumda 3, ortalamada 1 çevrimdir.\par
Hesap modülü atama aşamasında işlem koduna göre hesaplama birimlerinden biri hesaplamayı yapmak üzere seçilir ve gerekli giriş değerleri iletilir. Basit karşılaştırma devrelerinden oluşan aşama 1 çevrimde tamamlanır. Hesaplama aşaması ise her buyruk için farklı cevap süresine sahiptir. \par
Hesaplamanın bitmesi ile birlikte hesaplama modüllerinin çıkışında bulunan tampon belleklerde sonuçlar yazılmak üzere sıraya alınır. Geri yazma gerektirmeyen veya yanlış bir dallanma ile gelen buyruklar bu aşamada yazılmaz fakat işlem süresi olarak beklemek zorundadırlar. Geri yazma işlemi sonucu yazmaç öbeğine yazarken warp listesinde de buyruğun ait olduğu warp'un hazır bayrağını 1 yapar. Böylece aynı warp daha sonra tekrar seçilebilir. Geri yazma aşaması toplamda 1 çevrimde tamamlanır.\par

Sonuç olarak herhangi bir buyruğun boru hattını baştan sona tamamlaması için gerekli çevrim sayısı en kötü durum için Denklem \ref{equation:pipelineCycleEstimationWorstCase}'de belirtildiği şekilde, en iyi durum için Denklem \ref{equation:pipelineCycleEstimationBestCase}'de belirtildiği şekilde hesaplanabilir.\par

\begin{align} \label{equation:pipelineCycleEstimationWorstCase}
	T_{boru hatti} 	&= 1 + 35x(2N_{ada} + 1) + 1 + 3 + 1 + T_{hesap} + 1 \\
									&= T_{hesap} + 35 x (2N_{ada}+1) + 7
\end{align}

\begin{align} \label{equation:pipelineCycleEstimationBestCase}
	T_{boru hattı}  &= 1 + 2 + 1 + 3 + 1 + T_{hesap} + 1 \\
									&= T_{hesap} + 9
\end{align}

Her buyruk için en iyi durum ve en kötü durumda çevrim sayısı, 16 SIMD lane'den oluşan 4 ada için Şekil \ref{figure:bbc16c4a}'de sunulmuştur. Grafikte gösterilen çevrim sayıları, en kötü durumda buyruk başına boru hattının dolması için geçen süreyi ifade etmektedir. Boru hattının doldurulmasından sonra her çevrimde bir sonuç verilmesi beklendiğinden Şekil \ref{figure:bbc16c4a}'de sunulan değerler olabilecek en kötü sonuçlardır.\par

Mimaride ada sayısının artışı ile aynı anda çalışabilecek thread bloklarının sayısı, dolayısıyla paralellik artmaktadır. Ana bellek veri yolu genişliği sabit olup paralelleştirmede kısıtlayıcı etkendir. Mimaride eş zamanlı koşturulan toplam thread sayısının artırılması bellek işlemlerinde gecikmeyi artırır. En kötü durumda buyruk başına boru hattının ortalama dolma sürelerinin ada sayısına göre değişimi Şekil \ref{figure:bbc16xada}'de sunulmuştur.\par

\section{Tosun Alan Tüketimi Analizi}