\chapter{G\.{I}R\.{I}\c{S}}
Sayısal sinyal işleme algoritmalarında sıklıkla aynı işlem, farklı veriler üzerinde uygulanmaktadır. Geleneksel işlemcilerde bu tarz bir uygulama her veri için işlemin peşpeşe tekrarlanması ile gerçeklenir. Oysa ki algoritmaların bu özelliği, farklı veriler için uygulanacak aynı işlemin sırayla değil paralel çalıştırılması ile kayda değer performans artışlarını beraberinde getirir. Örneğin N elemanlı iki vektörün skalar çarpımı, N adet çarpma işleminden ve ardından N adet verinin toplanmasından oluşur. N adet çarpma işleminden herhangi birinin bir diğerini beklemeye ihtiyacı yoktur. Bu çarpma işlemlerinin peşi sıra yapıldığı ve paralel yapıldığı durumlar karşılaştırıldığında, paralel olan yöntemde N kata yakın performans artışı gözlenir.Paralelleştirmenin azımsanamayacak performans avantajından dolayı paralel çalışmayı destekleyecek donanım tasarımları üzerinde pek çok çalışma yapılmıştır. Literatürde öne çıkan çalışmaları 4 başlık altında toplamak mümkündür.  \par

Geleneksel işlemcilerde birden fazla iş parçacığının eş zamanlı çalıştırılabilmesini sağlamak amaçlı çok çekirdekli mimari tasarımları yaygın olarak kullanılmaktadır. Çok çekirdekli işlemcilerde bir çekirdek üzerinde 1 veya daha fazla thread koşturulması ile sinyal işleme fonksiyonlarında paralellik sağlanmaktadır. Endüstriyel uygulamalarda kullanılan DSP yongaları da çok çekirdekli işlemciler olarak düşünülebilir. Bu tarz mimarilerde çekirdekler programlanabilir olduğundan, yonganın fonksiyonel  
