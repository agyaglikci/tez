\chapter{G\.{I}R\.{I}\c{S}}
Sayısal sinyal işleme algoritmalarında tipik bir işlem, bir vektör ya da matrisin elemanları üzerinde ayrı ayrı çalıştırılmak suretiyle tekrarlanmaktadır. Algoritmaların bu özelliği, farklı veriler için uygulanacak aynı işlemin sırayla değil paralel çalıştırılması ile kayda değer performans artışlarını beraberinde getirir. Örneğin N elemanlı iki vektörün skalar çarpımı, N adet çarpma işleminden ve ardından N adet verinin toplanmasından oluşur. N adet çarpma işleminden herhangi birinin bir diğerini beklemeye ihtiyacı yoktur. Bu çarpma işlemlerinin peşi sıra yapıldığı ve paralel yapıldığı durumlar karşılaştırıldığında, paralel olan yöntemde N kata yakın performans artışı gözlenir.Paralelleştirmenin azımsanamayacak getirisinden dolayı paralel çalışmayı destekleyecek donanım tasarımları üzerinde pek çok çalışma yapılmış olup, literatürde öne çıkan çalışmalar 4 başlık altında toplanabilir.  \par

Çok çekirdekli işlemciler hali hazırda birden fazla iş parçacığının eş zamanlı çalışmasını desteklemektedir. Her bir çekirdek üzerinde 1 veya daha fazla thread koşturulması ile sinyal işleme fonksiyonlarında paralellik sağlanmaktadır. Endüstriyel uygulamalarda kullanılan DSP yongaları da çok çekirdekli işlemcilerdir. Bu tarz mimarilerde çekirdekler programlanabilir olduğundan geniş bir uygulama yelpazesine sahiptirler. Öte y
